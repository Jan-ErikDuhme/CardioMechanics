\section{File Formats}
\label{sec:fileformats}

\subsection{Input Files}
\label{subsec:infiles}

\subsubsection{CardioMechanics Config File (*.xml)}
\label{subsubsec:CardioMechanicsConfig}

This is the main configuration file that controls the mechanical simulations.
It uses the extensible markup language (XML) to store and define all simulation parameters.
The top level tags are defined as follows: \verb|General|, \verb|Mesh|, \verb|Materials|, \verb|Export|, \verb|Solver|, and \verb|Plugins|.
All available settings inside these tags are described in detail within the \nameref{sec:SimFramework}.
\begin{lstlisting}[language=XML,caption=CardioMechanics Config File (*.xml)]
<settings>A general description</settings>

<General>
    <LogFile> 	filename.txt 	</LogFile>
    <Verbose> 	false/true 	    </Verbose>
    <Debug>  	false/true 		</Debug>
</General>

<Mesh>
...
</Mesh>

<Materials>
...
</Materials>

<Export>
...
</Export>

<Solver>
...
</Solver>

<Plugins>
...
</Plugins>

\end{lstlisting}

\subsubsection{Element file (*.ele)}
\label{eleFile}

The element file contains only the elements for the mechanical mesh and has the following structure:
\begin{lstlisting}[language=Bash,caption=Element file (*.ele)]
    <# elements> <Nodes per Element> <# Attributes>
    <Element Index> <node 1> <node 2> ... <node n> <material index>
    ...
\end{lstlisting}
Only P1 and P2 elements are supported, hence each element needs either $n=4$ or $n=10$ nodes.

\subsubsection{Node File (*.node)}
\label{nodeFile}

The node file contains only the vertices for the mesh and has the following structure:
\begin{lstlisting}[language=Bash,caption=Node File (*.node)]
    <# nodes> <Dimension (must be 3)> <Boundary Conditions> (0: no boundary condition, 1: Dirichlet)> <BoundaryMarker (must be 0)>
    <Node Index> <x> <y> <z> <Boundary Condition>
    ...
\end{lstlisting}

The boundary condition is set as an integer number between 0 and 7. 
These numbers translated to their respective binary representation describe the Dirichlet boundary condition in the axes z,y,x, i.e. 001 (dec: 1) -> x-direction fixed, 010 (dec: 2) -> y-direction fixed, 100 (dec: 4) -> z-direction fixed, 111 (dec: 1+2+4 = 7) all directions fixed. 
Combinations of the aforementioned fixations are possible as well, i.e. you can fix a node in the x,y-plane by using the integer number 3 (binary: 011) etc.

\subsubsection{Surface Elements File (*.sur)}
\label{surFile}

This file contains a list of all surface triangle node indices of the mechanical mesh and its attributes.
\begin{lstlisting}[language=Bash,caption=Surface Elements File (*.sur)]
    <# elements> <Nodes per Element> <# Attributes (min 2; max 3)>
    <surface element Index> <node 1> <node 2> ... <node n> <material index> <surface index> <surface traction scaling>
    ...
\end{lstlisting}
Only P1 and P2 elements are supported, hence $n=3$ or $n=6$.
You need a minimum of two or a maximum of three attributes defined for each surface element.
Attribute 1 is the material index, which is used as a material tag during export.
Attribute 2 is the surface index, which is used by boundary conditions. 
Attribute 3 is optional and sets a scaling factor which can be applied to epicardial boundary conditions (\nameref{plugin:ContactHandling}, \nameref{plugin:Robin}, \nameref{plugin:RobinGeneral}).

\subsubsection{Fiber Orientation File (*.bases)}
\label{basesFile}

This file contains a list of all fiber orientations of the mechanical mesh. 
If you intend to use P2 elements, you have to submit the fiber orientations for each quadrature point as well.
Either way, the first orientation is always the centroid of the element.
This means it is possible to use a file for P2 elements with P1 elements as well, since the program will only read the number of quadrature points given in the file.
\begin{lstlisting}[language=Bash,caption=Fiber Orientation File (*.bases)]
    <# elements> <# qp>
    <element index> <fiber_x> <fiber_y> <fiber_z> <sheet_x> <sheet_y> <sheet_z> <normal_x> <normal_y> <normal_z>
    ...
\end{lstlisting}

\subsubsection{Local Activation Time File (*.txt)}
\label{LATFile}

This file contains activation times for each element of the mesh that defines the start of the contraction cycle in the tension model.
If a tension model requires a calcium transient, you need to make sure to provide one.
This is only relevant to the \autoref{tension:land17} model, which has multiple calcium transients to choose from already implemented.
Leave an empty line at the end of the file.
\begin{lstlisting}[language=Bash,caption=Local Activation Time File (*.txt)]
    <# elements> 1
    <element 1> <activation time in s>
    <element 2> <activation time in s>
    <element 3> <activation time in s>
    ...
    <element n> <activation time in s>
    
\end{lstlisting}

\subsection{Output Files}
\label{sec:outfiles}

\subsubsection{VTK Unstructured Grid Files (*.vtu)}
\label{subsubsec:VTK}

This file contains all data specified by the \nameref{subsec:Export} options and the \verb|SolverPlugin::Export()| function of each active plugin as long as the plugin supports exporting to VTK.
Refer to \autoref{tab:options} for an overview of export options to the VTK file type.
A file is written at each \verb|Export.TimeStep|.

\subsubsection{ParaView Data File (*.pvd)}

This file can be used to open, view, and post process the time series of \nameref{subsubsec:VTK} files in \href{https://www.paraview.org/}{ParaView}. 
Refer to the \href{https://www.paraview.org/Wiki/ParaView/Data_formats#PVD_File_Format}{ParaView Wiki on data formats} for more information.

\subsection{acCELLerate Specific Input Files}
\label{files:acCELLerate}

The input files listed in the following sections are specific to \ACC~and are only required if you intend to use the solver plugin \nameref{plugin:acCELLerate} to run electromechanically coupled simulations.

\subsubsection{Config File (*.aclt)}
\label{files:ACCConfig}

This is the main configuration file for EP tissue simulations.
Not all listed entries are needed for, e.g. Monodomain simulations, but possible entries are
\begin{lstlisting}[language=Bash,caption=acCELLerate configuration file (*.aclt)]
Resprefix      The prefix of the result data
Condition      Name of the condition file 
Sensor         Name of the sensor file 
CalcLength     Calculation length [s]
DTCell         Time step of cell model calculation [s]
DTExtra        Time step of extracellular calculation [s]
DTIntra        Time step of intracellular calculation [s]
BeginSave      Time at which saving of results starts [s]
DTSave         Time step of saving [s]
CellModelFile  Name of the cell model file
Material       Name of vector describing tissue classes
MaterialFibro  Name of vector describing fibroblast classes
MatrixIntra    Name of intracellular conductivity matrix
MatrixExtra    Name of extracellular conductivity matrix
MatrixExtraCombined       Name of extracellular conductivity matrix that is already combined with Intra (and Fibro)
MatrixFibro    Name of fibroblast conductivity matrix
BetaMyoFib     Number of Myo-Fibro gap junctions [1/m^3]
RMyoFib        Resistor of single myo-fibro gap junction [Ohm]
VolumeMyo      Relative amount of myocyte volume
VolumeExtra    Relative amount of extracellular volume
VolumeFibro    Relative amount of fibroblast volume
Domain         Type, either "Mono", "Bi", or "Tri"
Verbose        Print verbose information into terminal
Protocol       Name of verbose information file
HeteroFileIntra      Name of the heterogeneous cell model file (entries: <vector file name> <parameter name>)
HeteroFileFibro     Name of the heterogeneous fibro model file (entries: <vector file name> <parameter name>)
LoadBackup     Loads backup from standard file of <file> if declared
SaveBackup     Saves backup to standard file of <file> if declared
DTBackup       Time step of backup
Results        List of variables to be saved (e.g., Vm, Ve, AT, m, Na_i, I_Na, and any variable of the cell model)
Implicit       Number of Jacobi iterations for implicit calculation (!Only for Monodomain)
Force          Also calculate and output force
MassMatrixIntra        Name of intracellular mass matrix
ThetaIntra     Theta for intra PDE solver scheme. [0.0..1.0] 0=explicit, 1=implicit (default: 0 (FD) or 0.5 (FE))
MembraneCapacitance      Membrane capacitance per unit area, number or vector (F/m^2)
SurfaceToVolume       Myocyte surface to volume ratio, number or vector (1/m, default: cell model)
Gauss          Matrices for gauss interpolation and integration
ActivationThreshold       TMV threshold for activation time (default: 0V).
CurrentScheme  Godunow, SVI, ICI, None. SVI and None require Gauss matrices.
IntraIndexSet  PETSc IndexSet of the intracellular domain (if smaller than the xtracellular domain)
Compress       Compress output vectors using deflate/gzip.
NoExport       Deactivate Export
\end{lstlisting}

\subsubsection{Condition File (*.acnd)}
\label{files:condition}

This file contains the stimuli and initial/boundary conditions for an EP simulation.
\begin{lstlisting}[language=Bash,caption=acCELLerate condition file (*.acnd)]
    <node index> <amplitude> <cycle length> <duration> <temporal offset> Ii|Ie|If|Vm|Ve|Vf
    ...
\end{lstlisting}
The type of stimulus (last column) is given as a combination of first letter: either I(current) or V(voltage); domain of stimulus as second letter: intracellular=i, extracellular=e or fibroblastic=f.

\subsubsection{Cell Model File (*.cmf)}
\label{files:cellmodel}

Within the cell model file, the tissue classes from the geometry file are assigned to cell model types and force models.
\begin{lstlisting}[language=Bash,caption=acCELLerate cell model file (*.cmf)]
    <first tissue class> <elphpy model> <force model>
    <second tissue class> <elphpy model> <force model>
    <third tissue class> <elphpy model> <force model>
    ...
\end{lstlisting}
Elphy model and force model are given as the path to the .ev-file and .fv-file you want to use.

\subsubsection{Material File (*.def)}
\label{files:material}

Defines material properties, such as conductivities.
Each material that exists in the \emph{Material} array of the geometry has to be defined in this file with the following entries:
\begin{lstlisting}[language=Bash,caption=acCELLerate material file (*.def)]
[NAME MATERRIALNUMBER COLORCODE]
Kappa           FLOAT	FLOAT
AnisotropyX     FLOAT
AnisotropyY		FLOAT
AnisotropyZ		FLOAT
\end{lstlisting}
Kappa is the conductivity (second column, in S/m) at a given frequency (first column).
The COLORCODE value is currently not in use and can be set to an arbitrary number.
MATERIALNUMBER should be of type INT and should stay in the range 1 to 255 (same in the geometry array!).
You can use the anisotropy parameters to define different conductivites in fiber (AnisotropyX), sheet (AnisotropyY), and normal (AnisotropyZ) directions.
Kappa by itself defines the transversal conductivities, such that only AnisotropyX would need to be defined in case of transverse isotropy.
A second option to define a material is by using a definition of another material by referencing it with a hashtag followed by the material number if the conductivities are the same.
In the following example, we set a general material for the ventricles and atria to define the left/right atrium and ventricle.
\begin{lstlisting}[language=Bash,caption=acCELLerate material file example]
[Ventricle_myo 100 0]
Kappa           0	0.28
AnisotropyX     1.0
AnisotropyY		0.65
AnisotropyZ		0.35

[Atria_myo 200 0]
Kappa           0	0.1823
AnisotropyX     3.75

[Ventrikel_rechts 2 0]
#100

[Ventrikel_links 3 0]
#100

[right_atrium 32 0]
#200

[left_atrium 33 0]
#200
\end{lstlisting}

\subsubsection{Sensor File (*.txt)}
\label{files:Sensor}

The sensor files contain the information about the sensors that are placed in a simulated tissue. 
A sensor records any output parameter of the cell model in the applied node and saves the values in a textfile.
\begin{lstlisting}[language=Bash,caption=acCELLerate sensor file (*.txt)]
    <node index> <file name> <offset> <t_increment> Ii|Ie|If|Vm|Ve|Vf (or any output parameter of the applied cellmodel)
    ...
\end{lstlisting}

\subsubsection{CellModel Parameter File (*.ev)}
\label{files:ionmodel}

Cellmodel files define parameters and initial values for ionic models.
The general structure is given below.
Parameter files for all ionic models can be found in \href{https://github.com/KIT-IBT/CardioMechanics/tree/main/electrophysiology/data}{electrophysiology/data/}.

\begin{lstlisting}[language=Bash,caption=Ionic model parameter file (*.ev)]
CELLMODEL*IBT*KA
CELLMODELNAME
# of parameters to read
...
...
...
\end{lstlisting}

\subsubsection{ForceModel Parameter File (*.fv)}

Files with \emph{.fv} ending contain parameters and initial values for tension models.
Their structure is the same as for \emph{.ev} files described above.