\section{Getting Started}
\label{sec:gettingstrated}

\CM~is the main component of this software package.
To make electromechanical simulations possible, the additional components \ACC~and \CMT~(with subcomponents \EMT~and \FMT) were added and can be utilized via the plugin interface of \CM.
Currently, no precompiled version is available and the entire software has to be build from source.
Make sure to follow the build instructions  (\href{https://github.com/KIT-IBT/CardioMechanics/blob/main/docs/BUILD.md}{docs/BUILD.md}) to install from source using CMake or read the following section.

\subsection{System Requirements}

The following requirements have to be installed before trying to build CardioMechanics from source.
We recommend using a package manager (we use \href{https://www.macports.org}{macports} on our macOSX systems) whenever possible.
\begin{itemize}
    \item C and C++ compilers (e.g. gcc/g++ or clang/clang++)
    \item \href{https://cmake.org}{CMake}
    \item \href{https://zlib.net}{zlib}
    \item \href{https://gcc.gnu.org/fortran/}{gfortran}
    \item \href{https://git-scm.com}{git}
    \item \href{https://git-scm.com}{make}
    \item \href{https://www.mcs.anl.gov/petsc/}{PETSc}
    \item \href{https://vtk.org}{VTK}
    \item \href{https://www.open-mpi.org}{Open MPI}
    \item \href{https://www.python.org}{Python3} (optional, if you want to use some of the provided tools)
\end{itemize}
With \href{https://github.com/KIT-IBT/CardioMechanics/blob/main/installRequirements.sh}{installRequirements.sh} we provide a script to compile \href{https://www.open-mpi.org}{Open MPI}, \href{https://www.mcs.anl.gov/petsc/}{PETSc}, and \href{https://vtk.org}{VTK} from source with the most recently tested versions to ensure compatibility.
Building with \href{https://cmake.org}{CMake} as described in the next step requires the location and version of the tools as set in the script.
If you do want to use alternative locations/versions you have to link them as required.
By default, the script \href{https://github.com/KIT-IBT/CardioMechanics/blob/main/installRequirements.sh}{installRequirements.sh} compiles all components using a single process, which takes a considerable amount of time.
If you want to speed up this process, use the command \verb|make| with the option \verb|-j X| where X is the number of processes you want to use.
Additionally, adjust \verb|export AUTOMAKE_JOBS=X| specifically for \href{https://www.open-mpi.org}{Open MPI}.

\subsection{Installation Instructions}

Before continuing with compiling CardioMechanics using \href{https://cmake.org}{CMake}, add the following environmental variables to your systems configuration file ( e.g. .bashrc or .zshrc)
\begin{lstlisting}[language=bash]
export kaRootDir=$HOME/CardioMechanics
export THIRDPARTY_HOME=$kaRootDir/thirdparty
export PETSC_DIR=$THIRDPARTY_HOME/macosx
export PETSC_ARCH=petsc-v3.19.1
\end{lstlisting}
and add the location of the executables to your PATH variable (replace macosx with linux if you are on a linux machine).
\begin{lstlisting}[language=bash]
PATH="$PATH:$THIRDPARTY_HOME/macosx/openMPI-64bit/bin"
PATH="$PATH:$kaRootDir/_build/bin/macosx"
PATH="$PATH:$kaRootDir/tools/python"
export PATH
\end{lstlisting}
We assume here that you copied the repository to your \verb|$HOME| directory.
If you chose a different root directory, adjust the \verb|kaRootDir| variable accordingly.
If you used a different PETSc version in \href{https://github.com/KIT-IBT/CardioMechanics/blob/main/installRequirements.sh}{installRequirements.sh}, adjust the \verb|PETSC_ARCH| variable.
Now run 
\begin{lstlisting}[language=bash]
cmake -S . -B _build
\end{lstlisting}
to create the \verb|_build| folder and compile the code using
\begin{lstlisting}[language=bash]
cmake --build _build
\end{lstlisting}